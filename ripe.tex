\subsection{RIPE Atlas }

\subsubsection{DNS:}TESTAs reported in the previous study, the method used to conduct DNS hijacking is false resolutions. We encountered several domains for which the DNS resolutions returned a false redirect address, 10.10.34.36 instead of the domain’s correct IP address. \\
RIPE Atlas allowed us to set specific parameters for the DNS requests. First, we used a total of three different source probes from which our DNS requests were sent. Parties affiliated with RIPE Atlas operate the probes. Second, we specified the IP addresses of specific resolvers in a variety of locations within Iran. The constraints, from which we narrowed down our choice in resolvers by, include reliability and location. In total, the experiments used a series of three different probes and five DNS resolvers. \\
Our aim was to determine if the false resolution results vary depending on probe and/or DNS resolver. We used the top 40 most often accessed domains in Iran from Alexa as well as 20 additional domains previously reported to be victims of DNS hijacking.\\
	Our results revealed that there probe IP from which the DNS query was sent from had no effect on the accuracy of the result from any one resolver. That is to say, given two separate probes, resolving the same domain name at the same resolver, always received the same response. This suggests that all of the DNS requests were sent as is, and never tampered with on the way to the intended resolver. On the other hand, we determined that the resolver themselves were not centralized in terms of where they retrieved their rules from. In other words, the different resolvers could be inconsistent as we saw several cases of some resolvers returning false resolutions for the same domains that other resolvers returned accurate IP addresses for. As a result of this inconsistency, we compiled three categories in which our tested domains fall in.\\

\textbf{Always Resolves:} The domain names in this category were all accurately resolved. The resolutions were verified by through the use of scripts that ran the whois command to confirm the organization name matched the intended domain’s organization name. The initial tests executed 60 domain name resolution requests from three probes to three targeted resolvers. Thus each domain was tested from three probes, targeted at three solvers, for a total of 9 resolutions per domain. A domain was assigned to this category if all of the resolutions were accurate.

\textbf{ Resolves:} Our initial test of 60 domain names over three probes and three target resolvers revealed inconsistencies in resolutions. Specifically, of our first three resolvers, located in Tehran (2) and Isfahan, the results revealed that one resolver in Tehran, would consistently return accurate resolutions for domains which returned the 10.10.34.36 redirect pages by the other two resolvers. To determine whether or not this anomalous resolver was one of a kind, or if there exists true diversity in replies, depending on the chosen DNS resolver, we extended further testing. Our additional tests utilized the same three probes, with two additional resolvers in Tabriz and Zehadan. We tested every domain that returned the false redirect address in the previous round of resolutions, against these additional resolvers. We found further inconsistencies. The domain names in this category were accurately resolved by some resolvers and falsely resolved by others. This category contained a wide variety of domains in topics including: human rights, local blogs, national news and censorship circumvention.

\textbf{ Resolves:} The domain names in this category were never accurately resolved. This applies to all five resolvers. The category contained domains from topics including: political reform, gay+lesbian, international news, and western social media. The result returned from each DNS request in this category, was either no result at all, as a result of a timeout, or the redirect IP, 10.10.34.36.

Figure 1. Shows the accurate DNS resolution rate of a few of the queried domains


Figure 2. Shows the percentage accuracy per DNS resolver, of the domains that were blocked by at least one resolver (Percentage blocking per resolver location of “Bad” Domain Names).
