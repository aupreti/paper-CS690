\section{Background}\label{sec:background}

Over the past 15 years, the infrastructure of the Internet in Iran has been constructed in such way to give increasing amounts of control to the government. First, Internet exchange points (IXP) have been have been sponsored throughout the country. Internal IXPs increase connectivity and reliability, but also to decrease reliance on international infrastructure to maintain Internet communication. Furthermore, any Internet traffic exiting the country must cross one of few links, all of which are operated by the government, in order to interact with the global network outside of Iran. This single point of control gives the state operators additional control over the traffic allowed outside of the country. Whether the traffic is traversing international boundaries or is exclusively remaining within the country, many method are utilized to censor the content. \\
The government uses an array of mechanisms to control communications. These methods, detailed in the next section, include throttling connections based on protocols, filtering traffic based on HTTP headers, filtering traffic based on payload content and hijacking DNS requests. Lastly, Internet service providers are given monetary incentives to limit the connections they facilitate to only local domains. Government representatives have stated that the network has been constructed in such way to allow authorities to police Internet traffic in order to protect ``public morality" and discourage the ``dissemination" of lies ~\cite{anderson}. It must also be considered that the network has been constructed in such way to allow the complete shutdown of connections to the global community while maintaining Internet communications within Iran.
