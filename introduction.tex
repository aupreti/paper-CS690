\section{Introduction}\label{sec:intro}
Freedom House, the U.S. based, government-funded, non-partisan organization which researches the levels of democracy, political freedom and human rights in countries all over the world, ranked Iran as one of the worst countries in terms of free internet, second only to China. The government of Iran blocks social media and political and social content from a wide range of categories. 
	Over the past few years users of blocked domains in Iran have faced persecution and circumvention tools, which assisted civilians in accessing blocked domains, have been blocked through a variety of methods. The affects of internet censorship and lack of net freedom have been widely publicized. Notable cases include a complete shutdown in access to social media beginning during the 2012 presidential election campaigns in Iran, the 2014 arrests of Facebook activists who had used the social media website to motivate government reform, and unrelated arrests that same year of individuals who insulted the Supreme Leader, Imam Khomeni, over a text message application. 
