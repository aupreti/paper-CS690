\subsection{ICLab}
ICLab platform was used to conduct DNS, traceroute, TCP connect, TLS handshake and HTTP GET measurements. All except the traceroute experiments were successful. 
\subsubsection{DNS:} DNS resolution of each domain in our list was conducted from the Iranian VP using two resolvers: A local resolver of VP's choice and the google resolver, 8.8.8.8 were used. A python script was then used to compare the DNS responses from a US-based VP to the results obtained from Iran. \\
All queries using the google resolver yielded a null response. Some queries using the local resolver also yielded a null response even though a valid IP was obtained for the query using the US-based VP.  Figure x shows a comparison of DNS responses from a US VP and a Iranian VP.\\
\subsubsection{HTTP:}
We executed HTTP GET requests from sources in both the United States and Iran to use as control and experiment results, respectively. We analyzed the differences and compiled a list of inaccessible addresses by comparing the results from Iran to the results we received in the United States. Our original tests from the U.S.  resulted in a negligible number of failures to resolve which were of the form: HTTP 400 Bad Request or HTTP 404 Page Not Found. This occured in the case the request timed out or the domain name was intentional invalid. Invalid domains were used to test for HTTP request filtering by internet service providers (ISPs) in Iran. An invalid domain should return a 404 Page Not Found Error, however, the invlaid domains which were blocked resulted in 400 Bad Request or 403 Forbidden.\\ 

\subsubsection{Traceroute:} We setup our experiment to conduct traceroute measurements before we handed it over to the person responsible for ICLab. Our intention was to pinpoint the location of the censor within Iran’s network by using the traceroute emasurements. However, we got "traceroute not found or not installed" error for all of our measurements. We see this as one of the limitations of the ICLab platform compared to RIPE atlas. Researchers do not control the remote machines in the test countries so they lack a clear expectations of what kinds of tests will be successful. \\

