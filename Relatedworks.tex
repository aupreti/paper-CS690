\section{Related Works}\label{sec:relatedworks}

The previous experiments were coducted in 2012. These experiments pinpointed the methods used by the government to implement censorship as well as locating the sources of censorship. These experiments revealed many methods used by ISP and DNS resolvers in Iran in the effort to censor Internet users. Those methods include:

\begin{description}

\item[DNS Redirection:] The return of false, redirected pages in response to DNS requests for censored domain names. The previous experiments reported this redirect IP as 10.10.34.34.\\
\item[HTTP Filtering:] Access to specific websites blocked, based either on the content of the GET request itself, or the content in the response packet[s].\\
\item[Connection Throttling:] Iran reportedly throttles the connection of specific connections, namely those running over SSL or virtual private networks implemented using IPSec, among others methods of encryption. Furthermore, attempts to access specific websites, independent of protocol were reported to cause connection throttling.\\
\item[Broadband Speed Limitations:] The maximum bandwidth afforded to home user\’s is limited. This was previously believed to be enforced as a method of censorship, however more recent reports reveal this may be forced by limited network capabilities.\\

\end{description}

Our experiment simulates the HTTP filtering and DNS redirection described above in order to determine the consistencies of the filters and redirections. The previous experiment, conducted in 2012, was the first to pursue the technical aspects of the Internet censorship in Iran. Most of the research was done from a single vantage point and our experiment seeks to expand upon them.\\