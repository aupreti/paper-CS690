\begin{abstract}
We analyze Internet censorship in Iran through data collected using ICLab and RIPE Atlas. We use results from HTTP GET requests, TCP connect requests, TLS handshakes and DNS queries to confirm previous findings reported by Halderman et al. in 2013. Halderman et al. used a single VP in Iran whereas we use n different VPs to determine the consistency of the results from different vantage points. Through DNS queries conducted using network measurement platform, RIPE Atlas, we measure the scope of the censorship by probing 60 different domains. Through the HTTP GET requests, TCP connect requests and DNS queries conducted using ICLab, we measure the scope of the censorship by probing more than 500 domains. These domains include a test list for Iran compiled by citizen lab and top  websites for Iran reported by the Alexa website. We compare the results obtained through Iranian VPs with the results obtained from a VP located in the U.S. We conclude with the results confirming the blocked webpages either by way of HTTP filtering or DNS hijacking as previously reported by Halderman. In addition, we find that the censorship encountered at varying vantage points is consistent. In the process, we also discover some weaknesses of the ICLab platform and offer some recommendations.
\end{abstract}