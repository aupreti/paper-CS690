\begin{abstract}
We analyze Internet censorship in Iran through data collected using ICLab and RIPE Atlas. We use results from HTTP GET requests, DNS queries and traceroutes to confirm previous findings reported by Halderman et al and determine the consistency of the results using different vantage points. Through DNS queries conducted using network measurement platform, RIPE Atlas, we measure the scope of the censorship by probing 60 different domains. Through the HTTP GET requests, DNS queries and traceroutes conducted using ICLab, we measure the scope of the censorship by probing the top 500 domains as reported by the Alexa website as well as additional domains which were previously reported to be unresolvable. We conclude with the results confirming the blocked webpages either by way of HTTP filtering or DNS hijacking as previously reported by Halderman. In addition we find that the censorship encountered at varying vantage points is consistent. 
\end{abstract}